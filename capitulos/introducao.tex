% ----------------------------------------------------------
% Introdução
% ----------------------------------------------------------
\chapter{Introdução}

A introdução deve contextualizar o trabalho, apresentando o cenário em que o software será desenvolvido ou analisado. Descreva o problema que o projeto busca resolver, sua relevância e a justificativa para a escolha do tema. Explique por que essa solução é necessária e como ela pode impactar positivamente o contexto em que será aplicada. Além disso, forneça uma visão geral do que será abordado ao longo do relatório.

Exemplos de citações:

Citação com autor no texto, segundo \textcite{araujo2012} isso e tal.

Citação sem citar o autor no texto \cite{araujo2012}.




\section{Objetivos}

Esta seção apresenta o objetivo geral do estudo e os objetivos específicos, que detalham as etapas para sua realização.

\subsection{Geral}

Descreva o objetivo principal do projeto. Seja claro e direto, indicando o que se pretende alcançar com o projeto. Por exemplo, "Desenvolver e avaliar a usabilidade de um sistema de gerenciamento de tarefas para pequenas empresas."

\subsection{Específicos}

Liste os objetivos específicos que contribuirão para alcançar o objetivo geral. Eles devem ser detalhados e mensuráveis. Por exemplo, "Identificar as principais ferramentas para gestão de tarefas em pequenas empresas", "Identificar necessidades nas ferramentas atuais" e "Propor e desenvolver um protótipo de uma ferramenta fácil de usar para gestão de tarefas de pequenas empresas".
