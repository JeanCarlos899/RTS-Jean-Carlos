% ----------------------------------------------------------
% Fundamentação Teórica
% ----------------------------------------------------------
\chapter{Fundamentação Teórica}
\label{cha:fundamentacao_teorica}

Neste capítulo, são apresentados os conceitos essenciais que fundamentam o desenvolvimento deste trabalho. A primeira seção aborda o modelo \textit{Software as a Service} (SaaS), detalhando sua definição, arquitetura, vantagens e desafios. A segunda seção explora a aplicação de tecnologias no setor de transporte rodoviário, destacando como as plataformas SaaS podem otimizar a gestão e a eficiência operacional.

\section{Software as a Service (SaaS)}

O modelo \textit{Software as a Service} (SaaS), ou Software como Serviço, representa uma mudança de paradigma na forma como o software é distribuído e consumido. Em vez de adquirir licenças de uso perpétuo e instalar o software em servidores locais, os usuários acessam a aplicação pela internet, geralmente por meio de um navegador web, pagando uma taxa recorrente (assinatura) pelo serviço \cite{salesforce2025saas}.

Nesse modelo, toda a infraestrutura subjacente — servidores, armazenamento, redes e o próprio software — é gerenciada pelo provedor do serviço. Isso significa que o fornecedor é responsável pela manutenção, atualizações, segurança e disponibilidade da aplicação, permitindo que as empresas clientes foquem em suas atividades principais sem se preocupar com a complexidade da gestão de TI \cite{microsoft2025azure}.

\subsection{Arquitetura e Modelo de Negócio}

A arquitetura mais comum em soluções SaaS é a \textit{multi-tenancy} (multilocação), onde uma única instância da aplicação e da infraestrutura serve a múltiplos clientes (locatários ou \textit{tenants}). Embora compartilhem os mesmos recursos computacionais, os dados de cada cliente são mantidos isolados e seguros, garantindo a privacidade e a confidencialidade das informações \cite{microsoft2025learn}. Essa abordagem permite que o provedor otimize os recursos e reduza os custos, o que se reflete em preços mais acessíveis para o cliente final.

O modelo de negócio é baseado em assinaturas, que podem variar em preço conforme o número de usuários, os recursos contratados ou o volume de uso. Essa flexibilidade oferece escalabilidade, permitindo que as empresas ajustem o serviço de acordo com seu crescimento e suas necessidades, pagando apenas pelo que utilizam \cite{prologapp2024}.

\subsection{Vantagens e Desafios}

A adoção de plataformas SaaS oferece um conjunto significativo de vantagens para as empresas, especialmente para as de pequeno e médio porte, que podem não dispor de grandes orçamentos para investimentos em tecnologia. Entre os principais benefícios, destacam-se:

\begin{itemize}
    \item \textbf{Redução de Custos:} Elimina a necessidade de altos investimentos iniciais em hardware e licenças de software. Os custos de manutenção, atualização e suporte técnico também são responsabilidade do provedor \cite{prologapp2024}.
    \item \textbf{Acessibilidade e Mobilidade:} Por ser acessado via internet, o software pode ser utilizado de qualquer lugar e em diferentes dispositivos, bastando uma conexão ativa.
    \item \textbf{Escalabilidade:} As empresas podem facilmente aumentar ou diminuir a quantidade de recursos e usuários conforme a demanda, sem a necessidade de reestruturar a infraestrutura de TI.
    \item \textbf{Atualizações Automáticas:} O provedor é responsável por manter o software atualizado com as últimas funcionalidades e correções de segurança, garantindo que todos os clientes utilizem sempre a versão mais recente da aplicação \cite{gestran2025saas}.
    \item \textbf{Implementação Rápida:} A configuração e o início do uso de um sistema SaaS são geralmente mais rápidos em comparação com a instalação de um software tradicional (on-premise).
\end{itemize}

Apesar das vantagens, o modelo também apresenta desafios que devem ser considerados. A principal desvantagem é a dependência de uma conexão estável com a internet para acessar o serviço. Além disso, as opções de personalização podem ser mais limitadas em comparação com soluções desenvolvidas internamente, e a segurança dos dados, embora robusta, fica sob a responsabilidade de um terceiro, o que exige uma análise cuidadosa na escolha do fornecedor \cite{agendor2025b2b}.

\section{Tecnologia no Setor de Transporte Rodoviário}

O setor de transporte rodoviário de passageiros, historicamente caracterizado por processos manuais e uma gestão descentralizada, tem passado por uma profunda transformação impulsionada pela tecnologia. A digitalização das operações não apenas otimiza a gestão, mas também melhora a experiência do cliente e a segurança nas estradas.

Ferramentas como Sistemas de Gerenciamento de Transporte (TMS, \textit{Transportation Management System}), roteirizadores inteligentes e plataformas de venda online de passagens tornaram-se cruciais para a competitividade das empresas. Um TMS, por exemplo, centraliza informações sobre frotas, motoristas, rotas e finanças, permitindo um controle mais eficaz e uma tomada de decisão baseada em dados \cite{praxio2023tecnologia}.

\subsection{O Papel das Plataformas SaaS no Transporte}

As plataformas SaaS surgem como uma solução ideal para democratizar o acesso a essas tecnologias avançadas no setor de transporte alternativo. Ao oferecer um sistema robusto em um modelo de serviço, as barreiras de custo e complexidade técnica são drasticamente reduzidas.

Uma plataforma SaaS voltada para o transporte rodoviário, como a proposta neste trabalho, pode integrar diversas funcionalidades essenciais em um único ambiente. Segundo a Point Sistemas (2025), a aplicação desse modelo na logística permite obter visibilidade total da operação em tempo real, incluindo o acompanhamento de veículos, a gestão de ocorrências e a otimização de rotas com base em dados como geolocalização e tempo estimado de viagem \cite{pointsistemas2025logistica}.

A automação de tarefas, como a emissão de bilhetes, o controle de embarque e o fechamento financeiro, reduz a incidência de erros manuais e libera a equipe para se concentrar em atividades estratégicas. Além disso, a capacidade de integrar-se facilmente com outras ferramentas, como sistemas financeiros e de CRM, cria um ecossistema tecnológico coeso e eficiente, fundamental para a modernização e o crescimento sustentável do setor \cite{gestran2025saas}.