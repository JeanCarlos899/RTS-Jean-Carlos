\chapter{Fundamentação Teórica}\label{cha:fundamentacao_teorica}

Neste capítulo, são apresentados os conceitos essenciais que fundamentam o desenvolvimento deste trabalho. A primeira seção aborda o modelo \textit{Software as a Service} (SaaS), detalhando sua definição, arquitetura, vantagens e desafios. A segunda seção explora a aplicação de tecnologias no setor de transporte rodoviário, conectando a teoria à prática observada. Por fim, a terceira seção aprofunda-se na arquitetura e nas tecnologias específicas escolhidas para a construção do protótipo ViaBus.

\section{Software as a Service (SaaS)}

O modelo \textit{Software as a Service} (SaaS), ou Software como Serviço, representa uma mudança de paradigma na forma como o software é distribuído e consumido. Diferentemente do modelo tradicional \textit{on-premise}, onde o cliente adquire licenças e é responsável pela infraestrutura, no modelo SaaS o cliente paga uma assinatura periódica para acessar a aplicação pela internet, que é hospedada na nuvem pelo provedor do serviço \cite{moveideias2025saas}.

Nesse modelo, toda a infraestrutura subjacente — servidores, armazenamento, redes e o próprio software — é gerenciada pelo provedor. Isso significa que o fornecedor é responsável pela manutenção, atualizações, segurança e disponibilidade da aplicação, permitindo que as empresas clientes foquem em suas atividades principais sem se preocupar com a complexidade da gestão de TI \cite{moveideias2025saas}.

\subsection{Arquitetura e Modelo de Negócio}

A arquitetura mais comum em soluções SaaS é a \textit{multi-tenancy} (multilocação), onde uma única instância da aplicação e da infraestrutura serve a múltiplos clientes (locatários ou \textit{tenants}) \cite{frontegg2021multitenant}. Embora compartilhem os mesmos recursos computacionais, os dados de cada cliente são mantidos isolados e seguros, garantindo a privacidade e a confidencialidade das informações. Essa abordagem permite que o provedor otimize os recursos e reduza os custos, o que se reflete em preços mais acessíveis para o cliente final.

O modelo de negócio é baseado em assinaturas, que podem variar em preço conforme o número de usuários, os recursos contratados ou o volume de uso. Essa flexibilidade oferece escalabilidade, permitindo que as empresas ajustem o serviço de acordo com seu crescimento e suas necessidades, pagando apenas pelo que utilizam \cite{prologapp2024saas}.

\subsection{Vantagens e Desafios para PMEs}

A adoção de plataformas SaaS oferece um conjunto significativo de vantagens estratégicas para as Pequenas e Médias Empresas (PMEs). A principal delas é a drástica redução de custos, pois o modelo de assinatura elimina a necessidade de altos investimentos de capital (CAPEX) em licenças e hardware, transformando-os em despesas operacionais (OPEX) previsíveis \cite{praxio2021vantagens}.

Além da otimização financeira, outros benefícios se destacam. O modelo introduz uma notável escalabilidade e flexibilidade, permitindo que as empresas aumentem ou diminuam facilmente a capacidade de uso do software para responder rapidamente às mudanças do mercado. Soma-se a isso a acessibilidade e mobilidade intrínsecas ao serviço, que, por ser acessado via internet, pode ser utilizado de qualquer lugar e em diferentes dispositivos — algo transformador para o setor de logística. Ao terceirizar a gestão da infraestrutura de TI, as empresas também ganham maior foco no \textit{core business}, dedicando mais tempo e recursos à otimização de suas operações. Por fim, provedores de SaaS geralmente oferecem um nível de segurança e disponibilidade superior ao que uma PME poderia implementar por conta própria, garantido por Acordos de Nível de Serviço (SLAs) \cite{prologapp2024saas, praxio2021vantagens}.

Apesar das vantagens, o modelo também apresenta desafios, como a dependência de uma conexão estável com a internet e opções de personalização potencialmente mais limitadas. É justamente esse conjunto de benefícios, especialmente a redução de custos e a acessibilidade, que torna o modelo SaaS uma solução promissora para modernizar setores tradicionalmente analógicos, como o de transporte rodoviário.

\section{Tecnologia e Inovação no Transporte Rodoviário}

O setor de transporte rodoviário de passageiros, apesar de sua importância socioeconômica, apresenta um ritmo lento na adoção de tecnologias digitais \cite{sestsenat2021relatorio}. A ausência de soluções tecnológicas integradas, especialmente no segmento alternativo, resulta em ineficiências como a falta de controle sobre os processos, a incapacidade de atender a picos de demanda e altos índices de reclamação de clientes \cite{fateczl2022impactos}. Essa carência é corroborada pela pesquisa de mercado conduzida para este trabalho (ver \autoref{apendice:resultados}), que revelou que 100\% dos gestores entrevistados ainda dependem de métodos como cadernos de anotações e mensagens de WhatsApp para gerenciar suas operações.

Plataformas SaaS surgem como uma solução ideal para democratizar o acesso a tecnologias avançadas neste setor. A análise de soluções comerciais existentes, como TOTVS, iTransport e Praxio, revela uma lacuna de mercado: as ferramentas ou são muito complexas e caras para PMEs, ou são focadas em nichos específicos como o fretamento corporativo, não atendendo de forma integrada as necessidades do operador de transporte alternativo \cite{totvs2025passageiros, itransport2025gestao, praxioluna2025venda}.

\section{Arquitetura Tecnológica da Solução Proposta}

\subsection{Tecnologias do Backend}

O backend é o alicerce da plataforma, responsável pela lógica de negócios, persistência de dados e segurança. A \textit{stack} escolhida foi projetada para ser modular, robusta e escalável, utilizando tecnologias consolidadas no ecossistema TypeScript.

\subsubsection{NestJS e Arquitetura Modular}
O framework escolhido para o desenvolvimento do backend foi o NestJS. Trata-se de um framework Node.js progressivo, construído com e para o TypeScript, que utiliza uma arquitetura fortemente modular \cite{nestjs2025framework}. O pilar do NestJS é o conceito de Módulos, que organizam o código em blocos coesos e funcionais (e.g., um módulo para autenticação, outro para gestão de rotas). Cada módulo encapsula seus próprios \textit{controllers}, \textit{providers} (serviços) e pode importar ou exportar funcionalidades \cite{nestjs2025modules}. Essa estrutura promove uma forte separação de responsabilidades, facilita a reutilização de código e a manutenção do sistema, sendo ideal para a plataforma proposta, onde funcionalidades complexas podem ser desenvolvidas como módulos independentes \cite{devanddeliver2024architecture}.

\subsubsection{TypeORM e Mapeamento Objeto-Relacional (ORM)}
Para a camada de persistência de dados, a escolha foi o TypeORM. Um \textit{Object-Relational Mapper} (ORM) é uma técnica que cria uma ponte entre o paradigma orientado a objetos da aplicação e o paradigma relacional dos bancos de dados, permitindo que os desenvolvedores manipulem o banco de dados através de objetos e classes, abstraindo a necessidade de escrever consultas SQL manualmente \cite{logrocket2024typeorm}. O TypeORM é um ORM maduro para o ecossistema TypeScript, que utiliza intensivamente decoradores para definir Entidades (classes que mapeiam para tabelas) de forma declarativa e fortemente tipada. Isso acelera o desenvolvimento e reduz a probabilidade de erros relacionados a dados \cite{devto2024typeorm}.

\subsubsection{Autenticação com JWT e Passport.js}
A segurança da API é um requisito crítico. A estratégia de autenticação escolhida foi baseada em \textit{JSON Web Tokens} (JWT), implementada com a biblioteca Passport.js. JWT é um padrão aberto (RFC 7519) para a criação de tokens de acesso compactos e autossuficientes. Quando um cliente faz uma requisição a um recurso protegido, ele envia o JWT, e o servidor pode verificar a assinatura para autenticar o usuário sem precisar consultar um banco de dados de sessões, resultando em uma autenticação \textit{stateless} ideal para APIs RESTful \cite{soshace2024jwt}. O Passport.js atua como um \textit{middleware} de autenticação modular, e a estratégia \textit{passport-jwt} é projetada especificamente para extrair e verificar a validade desses tokens, oferecendo uma solução robusta e padronizada para proteger as rotas da API \cite{passportjs2025jwt}.

\subsection{Tecnologias do Frontend}

A interface do usuário (\textit{frontend}) é o principal ponto de contato com os gestores e passageiros. Sua arquitetura foi projetada com foco em performance e robustez, utilizando um ecossistema moderno baseado em React.

\subsubsection{Next.js, SSR e App Router}
O framework escolhido para o frontend é o Next.js. Uma de suas principais características é o suporte nativo à Renderização no Servidor (\textit{Server-Side Rendering} - SSR), na qual a página HTML é gerada no servidor a cada requisição. Isso melhora o tempo de carregamento inicial da página e a otimização para motores de busca (SEO) \cite{medium2025ssr}. Com a introdução do \textit{App Router}, o Next.js adotou por padrão o uso de \textit{React Server Components}, que permitem que a busca de dados e a renderização de componentes não interativos ocorram exclusivamente no servidor. Isso resulta em uma redução significativa da quantidade de JavaScript enviada ao cliente e otimiza a performance geral da aplicação \cite{nextjs2025servercomponents}.

\subsubsection{React, TypeScript e Ecossistema de UI}
A base da interface é o React, uma biblioteca JavaScript para a construção de interfaces baseada em um modelo de componentes reutilizáveis. Para garantir a robustez e a manutenibilidade, o React é utilizado com o TypeScript, um superconjunto do JavaScript que adiciona tipagem estática. O TypeScript permite a detecção de erros em tempo de compilação e serve como uma forma de documentação viva, tornando o código mais fácil de entender e facilitando a colaboração \cite{dhiwise2024reacttypescript}.

Para a estilização, foi adotado o Tailwind CSS, um framework \textit{utility-first} que acelera o desenvolvimento e garante consistência visual \cite{medium2025cssframeworks}. Comple-mentando-o, a biblioteca de componentes \textit{shadcn/ui} foi utilizada. Diferente de bibliotecas tradicionais, seus componentes são copiados para o código-fonte do projeto, dando ao desenvolvedor controle total sobre o código e permitindo personalizações profundas sem \textit{vendor lock-in} \cite{shadcnui2025docs}.

