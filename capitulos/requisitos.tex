\chapter{Requisitos do Sistema}\label{cha:requisitos}

Este capítulo apresenta a especificação dos requisitos funcionais e não funcionais do sistema ViaBus, obtidos através da pesquisa de mercado realizada e da experiência prática no setor de transporte de passageiros. Os requisitos foram organizados de forma a atender aos principais desafios operacionais identificados no levantamento de necessidades junto às empresas do setor.
\section{Requisitos Funcionais}

Os requisitos funcionais descrevem as funcionalidades que o sistema ViaBus deve executar. A \autoref{tab:requisitos-funcionais} apresenta os requisitos organizados por módulos funcionais.

\begin{table}[htbp]
  \centering
  \caption{Requisitos funcionais do sistema ViaBus}
  \label{tab:requisitos-funcionais}
  \resizebox{\textwidth}{!}{%
    \begin{tabular}{|c|l|p{8cm}|}
      \hline
      \textbf{ID} & \textbf{Módulo}      & \textbf{Descrição}                                     \\
      \hline
      \multicolumn{3}{|c|}{\textbf{Autenticação e Autorização}}                                   \\
      \hline
      RF01        & Registro             & Cadastro de usuários com validação de dados            \\
      RF02        & Login                & Autenticação segura via e-mail e senha                 \\
      RF03        & Permissões           & Controle de acesso com diferentes níveis de usuário    \\
      \hline
      \multicolumn{3}{|c|}{\textbf{Gestão Multi-empresa}}                                         \\
      \hline
      RF04        & Empresas             & Criação e gestão de perfis de empresa                  \\
      RF05        & Isolamento           & Separação completa de dados entre empresas             \\
      \hline
      \multicolumn{3}{|c|}{\textbf{Gestão de Recursos}}                                           \\
      \hline
      RF06        & Motoristas           & CRUD de motoristas com dados pessoais e profissionais  \\
      RF07        & Veículos             & CRUD de veículos com informações técnicas e manutenção \\
      RF08        & Paradas              & CRUD de pontos de parada com geolocalização            \\
      \hline
      \multicolumn{3}{|c|}{\textbf{Gestão de Rotas}}                                              \\
      \hline
      RF09        & Rotas                & Criação de rotas com sequência de paradas              \\
      RF10        & Horários             & Associação de horários e preços às rotas               \\
      RF11        & Mapas                & Visualização de rotas e paradas em mapas interativos   \\
      \hline
      \multicolumn{3}{|c|}{\textbf{Gestão de Viagens}}                                            \\
      \hline
      RF12        & Agendamento          & Criação de viagens baseadas em rotas e horários        \\
      RF13        & Atribuição           & Associação de veículos e motoristas às viagens         \\
      RF14        & Status               & Controle de status das viagens em tempo real           \\
      \hline
      \multicolumn{3}{|c|}{\textbf{Venda de Passagens}}                                           \\
      \hline
      RF15        & Interface Guiada     & Wizard para venda com fluxo orientado                  \\
      RF16        & Embarque/Desembarque & Seleção flexível de pontos na rota                     \\
      RF17        & Passageiros          & Cadastro completo dos dados dos passageiros            \\
      RF18        & Anti-Overbooking     & Verificação automática de disponibilidade              \\
      \hline
      \multicolumn{3}{|c|}{\textbf{Relatórios e Consultas}}                                       \\
      \hline
      RF19        & Listas               & Geração de listas de passageiros por viagem            \\
      RF20        & Busca                & Pesquisa avançada com múltiplos filtros                \\
      RF21        & Ocupação             & Visualização em tempo real da ocupação                 \\
      \hline
    \end{tabular}
  }
\end{table}

\section{Requisitos Não Funcionais}

Os requisitos não funcionais definem os critérios de qualidade e restrições técnicas que o sistema deve atender. A \autoref{tab:requisitos-nao-funcionais} apresenta esses requisitos organizados por categorias.

\begin{table}[htbp]
  \centering
  \caption{Requisitos não funcionais do sistema ViaBus}
  \label{tab:requisitos-nao-funcionais}
  \resizebox{\textwidth}{!}{%
    \begin{tabular}{|c|l|p{9cm}|}
      \hline
      \textbf{ID} & \textbf{Categoria} & \textbf{Descrição}                                                      \\
      \hline
      \multicolumn{3}{|c|}{\textbf{Usabilidade}}                                                                 \\
      \hline
      RNF01       & Responsividade     & Interface adaptável a diferentes dispositivos e tamanhos de tela        \\
      RNF02       & Navegação          & Interface intuitiva com navegação clara e feedback visual adequado      \\
      RNF03       & Acessibilidade     & Formulários guiados com indicadores de progresso em processos complexos \\
      \hline
      \multicolumn{3}{|c|}{\textbf{Segurança}}                                                                   \\
      \hline
      RNF04       & Autenticação       & Sistema de login seguro com controle de sessões                         \\
      RNF05       & Autorização        & Controle de acesso baseado em perfis de usuário                         \\
      RNF06       & Isolamento         & Separação total de dados entre diferentes empresas                      \\
      RNF07       & Validação          & Validação rigorosa de todas as entradas de dados                        \\
      \hline
      \multicolumn{3}{|c|}{\textbf{Performance}}                                                                 \\
      \hline
      RNF08       & Carregamento       & Tempo de resposta adequado para carregamento de páginas                 \\
      RNF09       & Consultas          & Otimização de consultas ao banco de dados                               \\
      RNF10       & Escalabilidade     & Suporte a crescimento do volume de dados e usuários                     \\
      \hline
      \multicolumn{3}{|c|}{\textbf{Manutenibilidade}}                                                            \\
      \hline
      RNF11       & Arquitetura        & Código organizado em módulos bem definidos                              \\
      RNF12       & Padrões            & Seguimento de padrões de codificação estabelecidos                      \\
      RNF13       & Documentação       & API e código adequadamente documentados                                 \\
      \hline
      \multicolumn{3}{|c|}{\textbf{Portabilidade}}                                                               \\
      \hline
      RNF14       & Containerização    & Sistema deployável em containers para diferentes ambientes              \\
      RNF15       & Configuração       & Suporte a múltiplos ambientes (desenvolvimento, teste, produção)        \\
      \hline
    \end{tabular}
  }
\end{table}