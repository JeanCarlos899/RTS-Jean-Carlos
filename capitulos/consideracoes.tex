\chapter{Considerações Finais}
\label{cha:consideracoes_finais}

Este trabalho de conclusão de curso abordou o desafio da gestão operacional no setor de transporte rodoviário alternativo de passageiros, um segmento caracterizado pela carência de soluções tecnológicas acessíveis. O problema central identificado foi a dependência de métodos manuais e fragmentados, que resultam em ineficiências e risco de erros. Diante deste cenário, o objetivo geral do projeto foi desenvolver um protótipo funcional de uma plataforma web, no modelo \textit{Software as a Service} (SaaS), denominada ViaBus, para centralizar e digitalizar essa gestão.

\section{Conclusão}

Conclui-se que o objetivo geral do trabalho foi atingido, uma vez que o protótipo funcional foi desenvolvido e sua arquitetura, detalhada. A implementação materializou os requisitos essenciais levantados na pesquisa de mercado — como o gerenciamento de recursos e a execução de operações de venda —, demonstrando que a abordagem técnica proposta é uma solução plausível para o problema. As contribuições deste projeto são, portanto, de ordem prática e documental: o próprio protótipo funcional, que serve como prova de conceito, e a documentação de uma proposta de arquitetura para a construção de um sistema SaaS multi-tenant focado neste nicho de mercado.

\section{Limitações do Trabalho}

É fundamental reconhecer as limitações inerentes a este trabalho para contextualizar o alcance de suas conclusões. A principal limitação metodológica reside na pesquisa de mercado, conduzida com uma amostra de apenas dois gestores, o que não permite a generalização estatística dos achados. No que tange à validação, a utilização de uma Avaliação Heurística em vez de testes com usuários finais oferece conclusões de usabilidade de caráter preliminar. Por fim, o escopo do protótipo funcional foi deliberadamente focado na perspectiva do gestor, não incluindo o desenvolvimento de uma aplicação para o passageiro, funcionalidade apontada como relevante na pesquisa.

\section{Trabalhos Futuros}

Com base nas limitações apontadas, delineia-se um caminho claro para a evolução do projeto. Recomenda-se, primeiramente, a validação da solução em maior escala, através de uma pesquisa de mercado quantitativa e da condução de testes de usabilidade com usuários reais. Em paralelo, o desenvolvimento do protótipo pode avançar com a implementação de funcionalidades de relatório (RF11) e do manifesto de viagem para o motorista (RF09). A criação de um aplicativo móvel complementar para o passageiro, permitindo a compra de bilhetes e o rastreamento de viagens, e a integração com sistemas de pagamento online representam os passos subsequentes para transformar o protótipo em um produto de mercado.