%% Configuração das Citações

%% Estilo
%\usepackage[num]{abntex2cite}			% Citações numéricas
% \usepackage[abnt-nbr10520=1988, alf, 
% abnt-emphasize = bf, 
% abnt-etal-list = 3,
% abnt-etal-text = it, 
% abnt-and-type = &, 
% abnt-last-names = abnt, 
% abnt-repeated-author-omit = no]{abntex2cite}			% Citações "AUTOR, ano"


% Definição do negrito

%% Colocar entre parênteses ou colchetes?
%% Padrão: Parênteses
%% * Fica mais agradável usar colchetes quando se usa citações numéricas
%\citebrackets[]							% Comente essa linha e o documento usará parênteses


%% Configura o "Citado nas Páginas ..." nas referências
%% Não mexa nesse:
% \renewcommand{\backref}{}

%% Esse é o texto do "Citado nas páginas ..."
% \renewcommand*{\backrefalt}[4]{
% 	\ifcase #1
% 		Nenhuma citação no texto.
% 	\or
% 		Citado na página #2.
% 	\else
% 		Citado #1 vezes nas páginas #2.
% 	\fi}

% ----------------------------------------------------------
% Referências e Citações no padrão da ABNT 2023
% ----------------------------------------------------------

\usepackage[style=abnt,
    backend=biber,
    maxbibnames=99,
    mincitenames=1,
    maxcitenames=3,
    backref=true,
    hyperref=true,
    giveninits=false,
    uniquename=false,
    uniquelist=false]{biblatex}

% Modificar a formatação do título do artigo para negrito
% \DeclareFieldFormat[article]{title}{\mkbibbold{#1}}

% Modificar a formatação do sobrenome em minúsculas
\renewcommand*{\mkbibnamefamily}[1]{#1}%

\addbibresource{bibliografia.bib}