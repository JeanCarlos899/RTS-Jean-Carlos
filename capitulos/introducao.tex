\chapter{Introdução}\label{cha:introducao}

Em um país de dimensões continentais como o Brasil, o transporte rodoviário de passageiros desempenha um papel estratégico na promoção da mobilidade interurbana, conectando municípios e regiões e garantindo o acesso a serviços, trabalho e lazer para milhões de pessoas. Esse serviço, regulamentado pela Agência Nacional de Transportes Terrestres (ANTT), é operado por empresas privadas e possui alta capilaridade, conectando centros urbanos de diferentes portes e promovendo interações espaciais ao longo do território nacional \cite{santos2024}.

A relevância desse modal de transporte fica evidente nos números: em 2019, o sistema interestadual atendeu mais de 2.000 municípios em 25 estados e no Distrito Federal, transportando cerca de 80 milhões de passageiros \cite{santos2024}. Esse volume expressivo de deslocamentos ressalta a necessidade de soluções tecnológicas para modernizar a gestão desse sistema, otimizando a venda de passagens e a administração operacional das empresas de transporte.

Diante desse cenário, este trabalho propõe o desenvolvimento de uma plataforma baseada no modelo Software como Serviço (\textit{SaaS, Software as a Service}) para a gestão integrada de empresas de transporte rodoviário alternativo, abordando seus impactos na eficiência operacional e na digitalização do setor. Segundo Chong e Carraro (2006, apud Melo et al., 2007), o \textit{SaaS} pode ser definido como "Software implementado como um serviço hospedado e acessado pela Internet" \cite{melo2007software}. Esse modelo permite que empresas utilizem soluções baseadas em nuvem, reduzindo custos operacionais e facilitando a escalabilidade dos serviços, o que pode ser especialmente vantajoso no setor de transporte rodoviário alternativo.

O transporte rodoviário intermunicipal de passageiros no Brasil enfrenta desafios significativos relacionados à adoção e integração de tecnologias avançadas. A ausência de sistemas integrados de gestão e a resistência à inovação tecnológica comprometem a eficiência operacional e a qualidade dos serviços prestados. De acordo com o Relatório de Análise de Impacto Regulatório da Agência Nacional de Transportes Terrestres (ANTT), a incerteza regulatória e a falta de experimentação com novas tecnologias dificultam a modernização do setor \cite{antt2022}.

\section{Justificativa}

A modernização do transporte rodoviário alternativo de passageiros no Brasil é essencial para aumentar a eficiência operacional das empresas do setor, garantindo maior acessibilidade e conectividade. A falta de soluções tecnológicas integradas tem limitado o crescimento desse segmento, que opera de forma fragmentada e sem padronização na gestão de frotas e vendas de passagens. A adoção de plataformas \textit{SaaS} pode transformar esse cenário ao digitalizar processos administrativos e operacionais, reduzindo custos e otimizando recursos. Estudos indicam que tecnologias no setor de transportes podem elevar a eficiência logística em até 15\%, diminuindo desperdícios e melhorando os serviços prestados \cite{setcepar2023}.

Além disso, uma plataforma \textit{SaaS} voltada para o transporte alternativo pode facilitar o acesso à inovação para pequenas e médias empresas, sem exigir altos investimentos. Esse modelo permite automação da bilhetagem, gestão integrada de rotas e melhor experiência para o passageiro. Essas soluções já demonstram impacto positivo em outros segmentos do transporte, aumentando a competitividade e garantindo serviços mais confiáveis \cite{prologapp2024}. Assim, a digitalização do setor não só fortalece as empresas, mas também melhora a mobilidade interurbana, tornando os serviços mais eficientes e acessíveis.

\section{Objetivos}

\subsection{Geral}

Desenvolver um protótipo funcional de uma plataforma SaaS para a gestão integrada do transporte rodoviário alternativo de passageiros, verificando a implementação de suas funcionalidades essenciais e avaliando a usabilidade de sua interface a partir de princípios de design.

\subsection{Específicos}

\begin{itemize}
    \item Realizar uma pesquisa de mercado para identificar as dores e os processos manuais de empresas do setor de transporte alternativo;

    \item Especificar os requisitos funcionais e não funcionais de um sistema de gestão com base nas necessidades levantadas na pesquisa;

    \item Desenvolver um protótipo funcional da plataforma ViaBus, implementando os módulos de gestão de rotas, paradas, veículos, motoristas e um fluxo para agendamento de passagens;

    \item Realizar uma verificação técnica do protótipo para analisar a aderência do software aos requisitos especificados e conduzir uma avaliação heurística para identificar possíveis melhorias de usabilidade na interface.
\end{itemize}

\section{Metodologia}

A metodologia empregada para a concepção e desenvolvimento do sistema ViaBus foi estruturada em três etapas sequenciais: levantamento de requisitos, desenvolvimento do protótipo e avaliação da solução.

A primeira etapa, de levantamento de requisitos, utilizou uma abordagem mista, partindo da observação empírica do autor sobre os desafios do setor de transporte alternativo. Tais observações foram então validadas por meio de uma pesquisa de mercado qualitativa. Para tal, foi elaborado um questionário online (detalhado no \autoref{apendice:questionario}) e aplicado junto a gestores de duas empresas do setor. As respostas, consolidadas na tabela do \autoref{apendice:resultados}, foram essenciais para a especificação formal dos requisitos que nortearam o desenvolvimento.

A segunda etapa, de desenvolvimento do protótipo, seguiu um processo iterativo alinhado a práticas de prototipagem evolutiva, onde o software é construído e refinado em ciclos contínuos \cite{sommerville2011software}. Adotou-se a estratégia \textit{Frontend-First}, que prioriza a construção da interface do usuário (UI) como guia para o desenvolvimento do sistema. Utilizando o framework Next.js e a biblioteca de componentes shadcn/ui, todas as telas da aplicação foram inicialmente desenvolvidas com dados estáticos (\textit{mockados}), permitindo a validação dos fluxos de interação antes da implementação da lógica de negócio no backend.

Por fim, a terceira etapa consistiu na avaliação do protótipo. Devido à impossibilidade de realizar testes com usuários finais, optou-se por uma verificação interna em duas frentes. A primeira foi uma \textit{Verificação de Requisitos}, na qual se analisou sistematicamente o atendimento aos requisitos funcionais e não funcionais. A segunda foi uma \textit{Avaliação Heurística}, um método de inspeção consolidado para encontrar problemas de usabilidade em interfaces \cite{Nielsen1994}. Atuando como avaliador especialista, o autor inspecionou o sistema com base nas 10 heurísticas de usabilidade de Nielsen, cujos resultados detalhados são apresentados no Capítulo 5.