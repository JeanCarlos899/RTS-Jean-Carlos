% ----------------------------------------------------------
% Introdução
% ----------------------------------------------------------
\chapter{Introdução}  

% Contextualização 

Em um país de dimensões continentais como o Brasil, o transporte rodoviário de passageiros desempenha um papel estratégico na promoção da mobilidade interurbana, conectando municípios e regiões e garantindo o acesso a serviços, trabalho e lazer para milhões de pessoas. Esse serviço, regulamentado pela Agência Nacional de Transportes Terrestres (ANTT), é operado por empresas privadas e possui alta capilaridade, conectando centros urbanos de diferentes portes e promovendo interações espaciais ao longo do território nacional \cite{santos2024}.

A relevância desse modal de transporte fica evidente nos números: em 2019, o sistema interestadual atendeu mais de 2.000 municípios em 25 estados e no Distrito Federal, transportando cerca de 80 milhões de passageiros \cite{santos2024}. Esse volume expressivo de deslocamentos ressalta a necessidade de soluções tecnológicas para modernizar a gestão desse sistema, otimizando a venda de passagens e a administração operacional das empresas de transporte.

% Delimitação do tema

Diante desse cenário, este trabalho propõe o desenvolvimento de uma plataforma baseada no modelo Software como Serviço (\textit{SaaS, Software as a Service}) para a gestão integrada de empresas de transporte rodoviário alternativo, abordando seus impactos na eficiência operacional e na digitalização do setor. Segundo Chong e Carraro (2006, apud Melo et al., 2007), o \textit{SaaS} pode ser definido como "Software implementado como um serviço hospedado e acessado pela Internet" \cite{melo2007software}. Esse modelo permite que empresas utilizem soluções baseadas em nuvem, reduzindo custos operacionais e facilitando a escalabilidade dos serviços, o que pode ser especialmente vantajoso no setor de transporte rodoviário alternativo.

% % Problema

O transporte rodoviário intermunicipal de passageiros no Brasil enfrenta desafios significativos relacionados à adoção e integração de tecnologias avançadas. A ausência de sistemas integrados de gestão e a resistência à inovação tecnológica comprometem a eficiência operacional e a qualidade dos serviços prestados. De acordo com o Relatório de Análise de Impacto Regulatório da Agência Nacional de Transportes Terrestres (ANTT), a incerteza regulatória e a falta de experimentação com novas tecnologias dificultam a modernização do setor \cite{antt2022}. 

% % Relevância do tema {Justificativa}
\section{Justificativa}

A modernização do transporte rodoviário alternativo de passageiros no Brasil é essencial para aumentar a eficiência operacional das empresas do setor, garantindo maior acessibilidade e conectividade. A falta de soluções tecnológicas integradas tem limitado o crescimento desse segmento, que opera de forma fragmentada e sem padronização na gestão de frotas e vendas de passagens. A adoção de plataformas \textit{SaaS} pode transformar esse cenário ao digitalizar processos administrativos e operacionais, reduzindo custos e otimizando recursos. Estudos indicam que tecnologias no setor de transportes podem elevar a eficiência logística em até 15\%, diminuindo desperdícios e melhorando os serviços prestados \cite{setcepar2023}.

Além disso, uma plataforma \textit{SaaS} voltada para o transporte alternativo pode facilitar o acesso à inovação para pequenas e médias empresas, sem exigir altos investimentos. Esse modelo permite automação da bilhetagem, gestão integrada de rotas e melhor experiência para o passageiro. Essas soluções já demonstram impacto positivo em outros segmentos do transporte, aumentando a competitividade e garantindo serviços mais confiáveis \cite{prologapp2024}. Assim, a digitalização do setor não só fortalece as empresas, mas também melhora a mobilidade interurbana, tornando os serviços mais eficientes e acessíveis.


\section{Objetivos}

\subsection{Geral}

Propor e desenvolver uma plataforma SaaS para a gestão integrada do transporte rodoviário alternativo de passageiros, avaliando sua funcionalidade e eficiência na otimização dos processos operacionais.

\subsection{Específicos}

\begin{itemize}
    \item Identificar as principais ferramentas tecnológicas utilizadas por empresas de transporte rodoviário alternativo, especialmente para gestão de paradas, rotas e agendamento de passagens;

    \item Levantar as necessidades e limitações dessas ferramentas, com foco nas dificuldades operacionais enfrentadas pelas empresas do setor;

    \item Propor e desenvolver um protótipo simplificado de sistema \textit{SaaS}, oferecendo funcionalidades para gestão integrada de paradas e rotas, além do agendamento digital de passagens;

    \item Realizar uma avaliação funcional e de usabilidade do protótipo desenvolvido, analisando sua eficiência e capacidade de atender às necessidades específicas das empresas do transporte alternativo.
\end{itemize}
