\chapter{Requisitos do Sistema}\label{cha:requisitos}

A concepção do sistema ViaBus partiu de uma necessidade concreta, identificada na pesquisa de mercado (\autoref{apendice:resultados}) realizada com gestores de empresas de transporte de passageiros. A pesquisa revelou um cenário operacional amplamente manual, onde ferramentas como \textit{"cadernos de anotações"} e \textit{"mensagens de WhatsApp"} são o padrão para tarefas críticas como a venda de passagens e a organização de listas de embarque. Essa dependência de métodos analógicos resulta nos principais desafios apontados pelos gestores: a dificuldade de \textit{"controlar os assentos já vendidos"}, a \textit{"perda de tempo organizando listas"} e a \textit{"comunicação com os motoristas"}.

Diante desse diagnóstico, os requisitos do ViaBus foram definidos para atacar diretamente essas dores. Eles representam a digitalização e a centralização de um processo hoje fragmentado e suscetível a erros. Este capítulo detalha esses requisitos em duas perspectivas: os requisitos funcionais, que descrevem o que o sistema deve ser capaz de fazer para o usuário, e os requisitos não funcionais, que definem os atributos de qualidade e as condições sob as quais o sistema deve operar.

\section{Requisitos Funcionais}

Os requisitos funcionais foram agrupados em categorias que espelham o fluxo de trabalho de uma empresa de transporte. Eles traduzem as necessidades operacionais em funcionalidades concretas, conforme detalhado na \autoref{tab:requisitos-funcionais-fechada}.

% --- Tabela de Requisitos Funcionais com Bordas ---
\begin{table}[htbp]
  \small
  \centering
  \caption{Requisitos funcionais do sistema ViaBus}
  \label{tab:requisitos-funcionais-fechada}
  \begin{tabular}{|p{1.5cm}|p{13.5cm}|}
    \hline
    \multicolumn{1}{|c|}{\textbf{ID}} & \multicolumn{1}{c|}{\textbf{Descrição da Funcionalidade}}                                                                                                     \\
    \hline
    \multicolumn{2}{|c|}{\small\bfseries Organização da Empresa e Recursos}                                                                                                                           \\
    \hline
    RF01                              & \RaggedRight O gestor deve poder cadastrar e gerenciar os dados da sua frota de veículos.                                                                     \\
    RF02                              & \RaggedRight O gestor deve poder cadastrar, consultar e atualizar as informações de seus motoristas, como dados de contato e validade da CNH.                 \\
    RF03                              & \RaggedRight O sistema deve permitir o cadastro de todos os pontos de parada (embarque/desembarque) utilizados pela empresa, incluindo sua localização.       \\
    RF04                              & \RaggedRight O gestor deve poder criar e organizar as rotas (itinerários) da empresa, definindo a sequência de paradas e os horários de operação.             \\
    \hline
    \multicolumn{2}{|c|}{\small\bfseries Operação e Venda de Passagens}                                                                                                                               \\
    \hline
    RF05                              & \RaggedRight O sistema deve permitir o agendamento de viagens futuras, associando uma rota, um veículo e um motorista.                                        \\
    RF06                              & \RaggedRight Um vendedor ou gestor deve poder vender passagens de forma rápida e guiada, selecionando a rota e a data da viagem.                              \\
    RF07                              & \RaggedRight Ao vender uma passagem, o sistema deve registrar as informações essenciais do passageiro.                                                        \\
    RF08                              & \RaggedRight O sistema não deve permitir, em hipótese alguma, a venda de uma passagem para um assento que já está ocupado em uma viagem (evitar overbooking). \\
    RF09                              & \RaggedRight O sistema deve gerar uma lista de passageiros de fácil consulta para cada viagem, que possa ser usada no momento do embarque.                    \\
    \hline
    \multicolumn{2}{|c|}{\small\bfseries Controle e Análise}                                                                                                                                          \\
    \hline
    RF10                              & \RaggedRight O gestor deve poder visualizar rapidamente a taxa de ocupação de cada viagem para tomar decisões.                                                \\
    RF11                              & \RaggedRight O sistema deve fornecer relatórios simples de vendas, mostrando o faturamento por rota ou por período.                                           \\
    \hline
  \end{tabular}
\end{table}

\section{Requisitos Não Funcionais}

Os requisitos não funcionais definem os critérios de qualidade que garantem uma boa experiência de uso e a confiabilidade do sistema. Eles são apresentados na \autoref{tab:requisitos-nao-funcionais-fechada}.

% --- Tabela de Requisitos Não Funcionais com Bordas ---
\begin{table}[htbp]
  \small
  \centering
  \caption{Requisitos não funcionais do sistema ViaBus}
  \label{tab:requisitos-nao-funcionais-fechada}
  \begin{tabular}{|p{1.5cm}|p{13.5cm}|}
    \hline
    \multicolumn{1}{|c|}{\textbf{ID}} & \multicolumn{1}{c|}{\textbf{Descrição da Funcionalidade}}                                                                                                     \\
    \hline
    \multicolumn{2}{|c|}{\small\bfseries Usabilidade e Acesso}                                                                                                                  \\
    \hline
    RNF01                             & \RaggedRight A interface do sistema deve ser clara, organizada e intuitiva, exigindo o mínimo de treinamento para um novo usuário.      \\
    RNF02                             & \RaggedRight O sistema deve ser fácil de usar tanto em um computador no escritório quanto em um celular ou tablet na rodoviária.        \\
    \hline
    \multicolumn{2}{|c|}{\small\bfseries Confiabilidade e Desempenho}                                                                                                           \\
    \hline
    RNF03                             & \RaggedRight O sistema deve ser rápido, com telas e informações carregando em poucos segundos.                                          \\
    RNF04                             & \RaggedRight A plataforma deve estar disponível para uso a maior parte do tempo, sem quedas ou instabilidades frequentes.               \\
    \hline
    \multicolumn{2}{|c|}{\small\bfseries Segurança}                                                                                                                             \\
    \hline
    RNF05                             & \RaggedRight O acesso ao sistema deve ser protegido por login e senha.                                                                  \\
    RNF06                             & \RaggedRight Os dados de uma empresa de transporte não podem, sob nenhuma circunstância, ser acessados pelos usuários de outra empresa. \\
    \hline
  \end{tabular}
\end{table}