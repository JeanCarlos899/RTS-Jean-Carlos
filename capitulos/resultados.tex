

\chapter{Resultados e Discussão}\label{cha:resultados}

\section{Verificação de Requisitos}

A primeira etapa da avaliação do protótipo consistiu na verificação sistemática do grau de implementação dos Requisitos Funcionais (RF) e Não Funcionais (RNF), conforme definidos no Capítulo~\ref{cha:requisitos}. Esta análise busca constatar de forma objetiva se as funcionalidades e características de qualidade planejadas foram efetivamente traduzidas no software desenvolvido. As Tabelas \ref{tab:verificacao-rf-nova} e \ref{tab:verificacao-rnf-nova} apresentam o resultado consolidado desta verificação.

\begin{table}[htbp]
  \small
  \centering
  \caption{Verificação dos Requisitos Funcionais (RF)}
  \label{tab:verificacao-rf-nova}
  \begin{tabular}{|p{1cm}|p{2.5cm}|p{11cm}|}
    \hline
    \textbf{ID} & \textbf{Status}             & \textbf{Observações e Evidências}                                                                                                                                                                                                                                                                 \\
    \hline
    RF01        & Implementado                & \RaggedRight O gestor pode cadastrar e gerenciar a frota de veículos através do módulo dedicado, conforme demonstrado na Figura~\ref{fig:veiculos}.                                                                                                                                               \\
    \hline
    RF02        & Implementado                & \RaggedRight O módulo de motoristas permite o CRUD completo das informações dos condutores (Figura~\ref{fig:motoristas}).                                                                                                                                                                         \\
    \hline
    RF03        & Implementado                & \RaggedRight O sistema permite o cadastro de pontos de parada, incluindo sua localização via mapa interativo (Figuras~\ref{fig:listagem-paradas} e \ref{fig:formulario-parada}).                                                                                                                  \\
    \hline
    RF04        & Implementado                & \RaggedRight A funcionalidade de criação e organização de rotas, com sequência de paradas, está implementada (Figura~\ref{fig:criacao-rota}).                                                                                                                                                     \\
    \hline
    RF05        & Implementado                & \RaggedRight O agendamento de viagens, associando rota, veículo e motorista, é funcional, como visto na Figura~\ref{fig:agendamento-viagem}.                                                                                                                                                      \\
    \hline
    RF06        & Implementado                & \RaggedRight O assistente (\textit{wizard}) de venda de passagens guia o usuário pelo processo de forma rápida, conforme demonstrado a partir da Figura~\ref{fig:wizard-rota}.                                                                                                                    \\
    \hline
    RF07        & Implementado                & \RaggedRight O formulário de venda de passagens coleta e armazena as informações essenciais do passageiro (Figura~\ref{fig:wizard-passageiro}).                                                                                                                                                   \\
    \hline
    RF08        & Parcialmente \-Implementado & \RaggedRight O sistema controla o número total de assentos disponíveis, impedindo a venda acima da capacidade (overbooking). A funcionalidade de gerenciamento de assentos específicos (mapa de assentos) foi definida como uma melhoria futura.                                                  \\
    \hline
    RF09        & Parcialmente \-Implementado & \RaggedRight A geração de listas de passageiros por viagem é funcional. No entanto, a opção de imprimir um manifesto e um módulo dedicado ao motorista para controle de embarque não foram contemplados no escopo deste protótipo.                                                                \\
    \hline
    RF10        & Parcialmente \-Implementado & \RaggedRight O sistema exibe o número de assentos livres, permitindo ao gestor inferir a ocupação. A visualização direta da "taxa de ocupação" em formato de dashboard (Figura~\ref{fig:dashboard}) é uma representação visual que, em futuras iterações, será conectada aos dados em tempo real. \\
    \hline
    RF11        & Não Implementado            & \RaggedRight A geração de relatórios financeiros foi deixada fora do escopo do protótipo devido a limitações de tempo, sendo um requisito prioritário para versões futuras do sistema.                                                                                                            \\
    \hline
  \end{tabular}
\end{table}

\begin{table}[htbp]
  \small
  \centering
  \caption{Verificação dos Requisitos Não Funcionais (RNF)}
  \label{tab:verificacao-rnf-nova}
  \begin{tabular}{|p{1.5cm}|p{3cm}|p{10cm}|}
    \hline
    \textbf{ID} & \textbf{Status} & \textbf{Observações e Evidências}                                                                                                                                                 \\
    \hline
    RNF01       & Implementado    & \RaggedRight A interface se mostrou clara e intuitiva nos principais fluxos de trabalho. A avaliação heurística, discutida na próxima seção, aponta pontos de melhoria pontuais.  \\
    \hline
    RNF02       & Implementado    & \RaggedRight O sistema foi desenvolvido com Tailwind CSS, garantindo a responsividade e a adaptação da interface a diferentes tamanhos de tela (desktop, tablet e mobile).        \\
    \hline
    RNF03       & Implementado    & \RaggedRight As tecnologias escolhidas (Next.js com SSR) resultaram em tempos de carregamento de página adequados para os fluxos de trabalho testados.                            \\
    \hline
    RNF04       & Implementado    & \RaggedRight A arquitetura de implantação no Google Cloud Run foi projetada para alta disponibilidade, embora testes de estresse em larga escala não tenham sido parte do escopo. \\
    \hline
    RNF05       & Implementado    & \RaggedRight O acesso é protegido por autenticação baseada em JWT, com rotas do backend e do frontend devidamente protegidas.                                                     \\
    \hline
    RNF06       & Implementado    & \RaggedRight A arquitetura multi-tenant com filtro automático por \textit{companyId} garante o isolamento lógico dos dados de cada empresa na camada de aplicação.                \\
    \hline
  \end{tabular}
\end{table}

\section{Avaliação Heurística da Interface}

Após a verificação dos requisitos, foi realizada uma Avaliação Heurística no protótipo do ViaBus, com o objetivo de identificar possíveis problemas de usabilidade na interface. Este método consiste na inspeção da interface por um avaliador, que a analisa com base em um conjunto de princípios e diretrizes de usabilidade reconhecidos, conhecidos como "heurísticas" \cite{Nielsen1994}. Para esta análise, foram utilizadas as 10 Heurísticas de Usabilidade de Jakob Nielsen, um dos conjuntos de princípios mais consolidados na área de Interação Humano-Computador.

A avaliação buscou identificar tanto os pontos fortes do \textit{design} proposto quanto as oportunidades de melhoria. Os resultados são discutidos a seguir, agrupados pelas heurísticas mais relevantes para o contexto do protótipo.

\subsubsection{Pontos Fortes da Interface}

O design do protótipo demonstrou forte aderência a diversas heurísticas fundamentais, resultando em uma base de interação sólida.

\begin{itemize}
  \item \textbf{Heurística 1: Visibilidade do status do sistema.} O sistema se comunica de forma eficaz com o usuário. Como visto na Figura~\ref{fig:wizard-rota}, o assistente de venda de passagens exibe claramente em qual das cinco etapas o usuário se encontra. Além disso, o uso consistente de \textit{toasts} de notificação (Figura~\ref{fig:tela-rotas}, por exemplo) para confirmar ações como "Rota criada com sucesso!" fornece feedback imediato e claro, mantendo o usuário informado sobre o resultado de suas operações.

  \item \textbf{Heurística 4: Consistência e padronização.} A interface do ViaBus mantém um alto grau de consistência visual e de interação. A estrutura de navegação, com o menu lateral (\textit{sidebar}) presente em todas as telas de gerenciamento (Figura~\ref{fig:dashboard}), e o padrão de tabelas com filtros e botões de ação (Figuras~\ref{fig:listagem-paradas} e \ref{fig:veiculos}) se repetem em todos os módulos. Essa padronização reduz a carga cognitiva do usuário, que não precisa reaprender a usar o sistema a cada nova tela.

  \item \textbf{Heurística 8: Estética e design minimalista.} A interface adota um design limpo e focado no conteúdo, evitando elementos visuais desnecessários que poderiam distrair o usuário. Os formulários, como o de criação de parada (Figura~\ref{fig:formulario-parada}), são um bom exemplo: os campos são claramente rotulados e organizados, apresentando apenas as informações essenciais para a tarefa em questão.
\end{itemize}

\subsubsection{Oportunidades de Melhoria}

A avaliação também identificou pontos onde a interface pode ser aprimorada para evitar erros e reduzir a fricção do usuário.

\begin{itemize}
  \item \textbf{Heurística 5: Prevenção de erros.} Embora o sistema valide os dados no \textit{backend}, a interface poderia ser mais proativa na prevenção de erros. Por exemplo, ao confirmar um agendamento de passagem (Figura~\ref{fig:agendamento-viagem}), o botão "Confirmar" não deve estar ativo se as informações como paradas de embarque e desembarque e preço total da passagem não forem informadas.

  \item \textbf{Heurística 6: Reconhecimento em vez de memorização.} Na interface de agendamento de viagem (Figura~\ref{fig:agendamento-viagem}), os campos para associar veículo e motorista são campos de busca. Embora funcional, o gestor precisa memorizar a placa do veículo ou o nome do motorista para encontrá-los. Uma melhoria significativa seria substituir esses campos por um seletor (\textit{dropdown}) ou uma busca com sugestões automáticas, que listaria os recursos disponíveis. Isso trocaria a necessidade de memorização pela simples tarefa de reconhecimento, tornando o processo mais rápido e menos propenso a erros.
\end{itemize}

Em suma, a avaliação heurística indica que o protótipo do ViaBus possui uma base de usabilidade sólida, com destaque para a clareza, consistência e design minimalista. As principais oportunidades de aprimoramento concentram-se na implementação de mecanismos para a prevenção de erros e na redução da carga de memorização do usuário em fluxos de trabalho específicos.
\section{Discussão dos Resultados}

A análise dos resultados indica que o protótipo do ViaBus atingiu seu objetivo primário: materializar as funcionalidades essenciais para a digitalização da gestão de transporte alternativo. A verificação de requisitos demonstrou que o núcleo operacional do sistema — abrangendo o gerenciamento de recursos como frotas e motoristas, e a execução de operações como agendamento de viagens e vendas — foi implementado com sucesso. Contudo, a análise também revelou que funcionalidades analíticas avançadas, como a de relatórios, foram designadas como trabalhos futuros devido a limitações no escopo do projeto. Do ponto de vista da usabilidade, a avaliação heurística revelou uma base de interface sólida e consistente, mas também apontou para a necessidade de refinar mecanismos de prevenção de erros para garantir uma operação mais segura no dia a dia.

A principal contribuição do protótipo reside na sua capacidade de atacar diretamente os desafios operacionais revelados pela pesquisa de mercado (ver \autoref{apendice:resultados}). A dependência de "cadernos de anotações" e processos manuais via WhatsApp, uma realidade para 100\% dos gestores entrevistados, é diretamente substituída pelo assistente de venda de passagens (Figuras \ref{fig:wizard-rota} a \ref{fig:wizard-confirmacao}), que centraliza e digitaliza o processo. O requisito RF08 (Anti-Overbooking), que foi implementado, soluciona uma das maiores preocupações operacionais — a venda de assentos acima da capacidade —, um problema recorrente no controle manual. Da mesma forma, a geração automática da lista de passageiros (RF09) elimina a "perda de tempo organizando listas", um dos principais entraves à eficiência citados na pesquisa.

Apesar dos avanços, é crucial reconhecer as limitações funcionais do protótipo em seu estado atual. A não implementação do módulo de relatórios financeiros (RF11) é a lacuna mais significativa, pois impede que o gestor extraia inteligência de negócio a partir dos dados coletados, uma das vantagens centrais prometidas pelo modelo SaaS. Adicionalmente, a implementação parcial do RF09, que não inclui um manifesto de viagem ou módulo dedicado ao motorista, significa que a "comunicação com os motoristas" — outro desafio identificado na pesquisa — ainda não está totalmente digitalizada no protótipo, resolvendo apenas parte do problema de comunicação entre a gestão e a operação de embarque.

A avaliação heurística reforça que a usabilidade da plataforma é um ponto forte, especialmente sua consistência e clareza (aderência às Heurísticas 4 e 8 de Nielsen), o que é vital para um público-alvo que pode não ter familiaridade com sistemas complexos. Contudo, a falha identificada na "Prevenção de Erros" (Heurística 5) — a ausência de confirmação antes de ações destrutivas — representa um risco operacional real. Em um ambiente de produção, um clique acidental poderia levar à exclusão de uma rota inteira, demonstrando que a robustez de um sistema não depende apenas de sua funcionalidade, mas também de mecanismos que protejam o usuário de erros inadvertidos.

Finalmente, ao contrastar o protótipo com as soluções de mercado citadas na Fundamentação Teórica, fica evidente o nicho que o ViaBus busca ocupar. Comparado a sistemas de grande porte, ele é funcionalmente mais enxuto. No entanto, sua proposta de valor não reside na competição direta de funcionalidades, mas na simplicidade, usabilidade e baixo custo de entrada, preenchendo a lacuna de mercado para pequenas e médias empresas de transporte alternativo que consideram as soluções existentes complexas ou com um custo proibitivo.