%% Abstract (configurado para língua inglesa)
\begin{resumo}[Abstract]    % Título do Resumo (Abstract = Resumo em inglês)
  \begin{otherlanguage*}{english}  % Língua do texto
  The alternative road passenger transport sector lacks accessible technological solutions, resulting in predominantly manual, fragmented, and inefficient management. This work addresses this problem through the design and development of a functional prototype of a web platform, named ViaBus, based on the Software as a Service (SaaS) model. The adopted methodology started with a qualitative market research for requirements gathering, followed by the modeling of a multi-tenant software architecture and the implementation of the prototype using NestJS for the backend and Next.js for the frontend. As a result, a functional application was delivered that centralizes the management of fleets, drivers, routes, and ticket sales, with the requirements verification confirming the implementation of the operational core. Additionally, a heuristic evaluation of the interface indicated a solid usability foundation, with strong adherence to principles of consistency and clarity, although it pointed out opportunities for improvement in error prevention. It is concluded that the SaaS approach focused on simplicity is a technically feasible solution for the studied market niche, and that the ViaBus prototype serves as an effective proof of concept for this proposal, with its limitations and opportunities for future work duly documented.
  
  \vspace{\onelineskip}
  \noindent
  \textbf{Keywords}: Passenger Transportation; Software as a Service; Management System; Software Architecture.
  \end{otherlanguage*}
  \end{resumo}