\chapter{Tecnologias Envolvidas} \label{cha:tecnologias}

A construção do protótipo ViaBus foi fundamentada em um conjunto de tecnologias de mercado, selecionadas a partir de critérios objetivos como adequação aos requisitos do projeto, ecossistema e características técnicas específicas. Este capítulo apresenta as principais ferramentas, bibliotecas e frameworks utilizados, organizados em tabelas por categoria para facilitar a consulta e a compreensão da arquitetura da solução. Cada tabela detalha a aplicação específica da tecnologia no projeto e a justificativa técnica para sua escolha. Para consultar as versões exatas de cada dependência utilizada, veja o \autoref{apendice:repositorios}, que lista os repositórios do código-fonte do projeto.

\section{Tecnologias do Backend}

As tecnologias do \textit{backend} formam o alicerce da aplicação, responsáveis pela lógica de negócio, segurança e persistência dos dados. As principais escolhas estão detalhadas na \autoref{tab:tecnologias-backend-dividida}.

\begin{table}[htbp]
  \small
  \centering
  \caption{Tecnologias utilizadas no Backend}
  \label{tab:tecnologias-backend-dividida}
  \begin{tabular}{|p{3cm}|p{6cm}|p{5.5cm}|}
    \hline
    \textbf{Tecnologia} & \textbf{Aplicação no Projeto ViaBus}                                                                                                        & \textbf{Justificativa da Escolha}                                                                                                                                               \\
    \hline
    NestJS              & \RaggedRight Framework principal para a construção da API RESTful, estruturando os módulos de negócio (rotas, motoristas, passagens, etc.). & \RaggedRight A arquitetura modular e o sistema de injeção de dependência nativo facilitam a organização e a manutenção de um projeto com múltiplos domínios.                    \\
    \hline
    TypeORM             & \RaggedRight Mapeamento das classes do domínio para as tabelas do banco de dados PostgreSQL, abstraindo a escrita de consultas SQL.         & \RaggedRight Integração madura com o ecossistema TypeScript e suporte a \textit{decorators} para a definição de entidades, o que aumenta a produtividade e a clareza do código. \\
    \hline
    Passport.js com JWT & \RaggedRight Implementação da estratégia de autenticação e autorização, protegendo as rotas da API e gerenciando as sessões dos usuários.   & \RaggedRight Padrão de mercado para autenticação \textit{stateless} em APIs. A modularidade do Passport.js permite uma implementação desacoplada e segura.                      \\
    \hline
  \end{tabular}
\end{table}

\section{Tecnologias do Frontend}

O \textit{frontend} é a camada de apresentação do sistema, com a qual o usuário interage diretamente. As tecnologias foram escolhidas com foco em performance e experiência de desenvolvimento, conforme a \autoref{tab:tecnologias-frontend-dividida}.

\begin{table}[htbp]
  \small
  \centering
  \caption{Tecnologias utilizadas no Frontend}
  \label{tab:tecnologias-frontend-dividida}
  \begin{tabular}{|p{3cm}|p{6cm}|p{5.5cm}|}
    \hline
    \textbf{Tecnologia} & \textbf{Aplicação no Projeto ViaBus}                                                                                                             & \textbf{Justificativa da Escolha}                                                                                                                                               \\
    \hline
    Next.js             & \RaggedRight Framework principal para a construção da interface de usuário (UI), incluindo o painel de gestão e os fluxos de venda de passagens. & \RaggedRight Suporte nativo a Server-Side Rendering (SSR) e Server Components, estratégias que melhoram o tempo de carregamento inicial e a performance percebida pelo usuário. \\
    \hline
    React               & \RaggedRight Biblioteca base para a criação de toda a interface em um modelo de componentes reutilizáveis (botões, formulários, tabelas, etc.).  & \RaggedRight Ecossistema consolidado e vasta disponibilidade de bibliotecas, o que viabilizou a rápida prototipação da interface.                                               \\
    \hline
    TypeScript          & \RaggedRight Utilizado em todo o desenvolvimento (backend e frontend) para adicionar tipagem estática ao JavaScript.                             & \RaggedRight Aumenta a manutenibilidade do código e reduz a ocorrência de erros em tempo de execução, servindo como uma forma de documentação para a estrutura de dados.        \\
    \hline
    Tailwind CSS        & \RaggedRight Framework CSS utilizado para a estilização de todos os componentes da interface.                                                    & \RaggedRight A abordagem \textit{utility-first} acelera o desenvolvimento da UI e facilita a manutenção de um design system consistente em toda a aplicação.                    \\
    \hline
  \end{tabular}
\end{table}

\section{Infraestrutura e Ferramentas de Suporte}

Este conjunto de ferramentas dá suporte ao desenvolvimento, à persistência dos dados e à implantação do sistema, garantindo consistência e qualidade ao longo do ciclo de vida do projeto (\autoref{tab:tecnologias-infra-dividida}).

\begin{table}[htbp]
  \small
  \centering
  \caption{Infraestrutura e Ferramentas de Suporte}
  \label{tab:tecnologias-infra-dividida}
  \begin{tabular}{|p{3cm}|p{6cm}|p{5.5cm}|}
    \hline
    \textbf{Tecnologia} & \textbf{Aplicação no Projeto ViaBus}                                                                                          & \textbf{Justificativa da Escolha}                                                                                                                    \\
    \hline
    PostgreSQL          & \RaggedRight Sistema de Gerenciamento de Banco de Dados (SGBD) relacional para a persistência de todos os dados da aplicação. & \RaggedRight SGBD de código aberto com funcionalidades avançadas, como suporte a dados geoespaciais (PostGIS), e alta compatibilidade com o TypeORM. \\
    \hline
    Docker              & \RaggedRight Conteinerização do backend, do frontend e do banco de dados para os ambientes de desenvolvimento e produção.     & \RaggedRight Garante a paridade entre os ambientes, simplifica a configuração do projeto e facilita a implantação em qualquer provedor de nuvem.     \\
    \hline
    ESLint e Prettier   & \RaggedRight Ferramentas para análise estática e formatação automática do código-fonte.                                       & \RaggedRight Asseguram a padronização e a qualidade do código, independentemente do desenvolvedor, facilitando a leitura e a manutenção futura.      \\
    \hline
  \end{tabular}
\end{table}