%% Resumo
\begin{resumo}
  O setor de transporte rodoviário alternativo de passageiros carece de soluções tecnológicas acessíveis, resultando em uma gestão predominantemente manual, fragmentada e ineficiente. Este trabalho aborda este problema através do projeto e desenvolvimento de um protótipo funcional de uma plataforma \textit{web}, denominada ViaBus, no modelo \textit{Software as a Service} (SaaS). A metodologia adotada partiu de uma pesquisa de mercado qualitativa para o levantamento de requisitos, seguida pela modelagem de uma arquitetura de \textit{software} \textit{multi-tenant} e a implementação do protótipo com as tecnologias NestJS para o \textit{backend} e Next.js para o \textit{frontend}. Como resultado, foi entregue uma aplicação funcional que centraliza o gerenciamento de frotas, motoristas, rotas e a venda de passagens, com a verificação de requisitos confirmando a implementação do núcleo operacional. Adicionalmente, uma avaliação heurística da interface indicou uma base de usabilidade sólida, com forte aderência a princípios de consistência e clareza, embora tenha apontado oportunidades de melhoria na prevenção de erros. Conclui-se que a abordagem SaaS com foco na simplicidade é uma solução tecnicamente viável para o nicho de mercado estudado, e que o protótipo ViaBus serve como uma prova de conceito eficaz dessa proposta, com suas limitações e oportunidades de trabalhos futuros devidamente documentadas.
  \vspace{\onelineskip}
  \noindent

  \textbf{Palavras-chave}: Transporte de Passageiros; Software como Serviço; Sistema de Gestão; Arquitetura de Software.
\end{resumo}