\chapter{Introdução}\label{cha:introducao}

Em um país de dimensões continentais, o transporte rodoviário de passageiros funciona como um elemento estratégico para a integração socioeconômica, conectando municípios e garantindo a mobilidade da população \cite{FGV2023}. Enquanto o sistema interestadual é regulamentado em nível federal pela Agência Nacional de Transportes Terrestres (ANTT), conforme estabelecido em sua lei de criação \cite{BRASIL2001}, uma vasta e heterogênea rede de transporte intermunicipal opera sob jurisdição estadual.

No estado do Piauí, esta modalidade é uma realidade consolidada, cuja organização é definida pela Lei Nº 8.562 de 2025, que dispõe sobre o Sistema de Transporte Rodoviário Intermunicipal de Passageiros do Estado do Piauí (STRIP/PI). A referida lei classifica o "serviço alternativo"\ como uma das categorias oficiais do sistema, designando a Secretaria dos Transportes (SETRANS) como o poder concedente \cite{PIAUI2025}. Este serviço, prestado por veículos de menor porte, surge para suprir as lacunas deixadas pelo sistema convencional. A existência de uma legislação específica demonstra a relevância do setor, mas também evidencia a necessidade de ferramentas de gestão que se adaptem às suas particularidades operacionais.

Diante desse cenário, este trabalho propõe o desenvolvimento de uma plataforma
baseada no modelo Software como Serviço (\textit{SaaS, Software as a Service})
para a gestão integrada de empresas de transporte rodoviário alternativo.
A escolha por este modelo se fundamenta na definição de Chong e Carraro
(2006 apud \textcite{melo2007software}) como um "software implementado
como um serviço hospedado e acessado pela Internet", o que permite que empresas
utilizem soluções baseadas em nuvem com custos operacionais reduzidos e alta
escalabilidade --- características especialmente vantajosas para o setor em foco.

Este cenário, onde um serviço regulamentado e essencial como o transporte alternativo ainda opera com uma gestão predominantemente analógica — fato constatado na pesquisa de mercado realizada para este trabalho (\autoref{apendice:resultados}) —, evidencia os desafios para a modernização do setor. A ausência de sistemas de gestão integrados e a consequente dependência de processos manuais não apenas comprometem a eficiência operacional e a qualidade dos serviços prestados, mas também criam uma barreira para a inovação e o crescimento sustentável das empresas que atuam nesse segmento.

\section{Justificativa}

O transporte rodoviário alternativo de passageiros no Brasil, embora essencial para a conectividade regional, opera de forma majoritariamente analógica e fragmentada. Uma pesquisa de mercado realizada para este estudo (\autoref{apendice:resultados}) revelou que a gestão de frotas e a venda de passagens ainda são fortemente dependentes de processos manuais, como o uso de cadernos de anotações e planilhas. Essa carência de ferramentas digitais integradas limita a eficiência e o potencial de crescimento do setor. Neste contexto, a adoção de plataformas \textit{SaaS} surge como uma solução estratégica para transformar esse cenário, otimizando recursos e reduzindo custos operacionais. Estudos de mercado mais amplos corroboram essa visão, indicando que tecnologias no setor de transportes podem elevar a eficiência logística em até 15\% \cite{setcepar2023}.

Além disso, uma plataforma \textit{SaaS} voltada para o transporte alternativo pode facilitar o acesso à inovação para pequenas e médias empresas, sem exigir altos investimentos. Esse modelo permite automação da bilhetagem, gestão integrada de rotas e melhor experiência para o passageiro. Essas soluções já demonstram impacto positivo em outros segmentos do transporte, aumentando a competitividade e garantindo serviços mais confiáveis \cite{prologapp2024}. Assim, a digitalização do setor não só fortalece as empresas, mas também melhora a mobilidade interurbana, tornando os serviços mais eficientes e acessíveis.

\section{Objetivos}

\subsection{Geral}

Desenvolver um protótipo funcional de uma plataforma SaaS para a gestão integrada do transporte rodoviário alternativo de passageiros, verificando a implementação de suas funcionalidades essenciais e avaliando a usabilidade de sua interface a partir de princípios de design.

\subsection{Específicos}

\begin{itemize}
    \item Realizar uma pesquisa de mercado para identificar as dores e os processos manuais de empresas do setor de transporte alternativo;

    \item Especificar os requisitos funcionais e não funcionais de um sistema de gestão com base nas necessidades levantadas na pesquisa;

    \item Desenvolver um protótipo funcional da plataforma ViaBus, implementando os módulos de gestão de rotas, paradas, veículos, motoristas e um fluxo para agendamento de passagens;

    \item Realizar uma verificação técnica do protótipo para analisar a aderência do software aos requisitos especificados e conduzir uma avaliação heurística para identificar possíveis melhorias de usabilidade na interface.
\end{itemize}

\section{Metodologia}

A metodologia empregada para a concepção e desenvolvimento do sistema ViaBus foi estruturada em três etapas sequenciais: levantamento de requisitos, desenvolvimento do protótipo e avaliação da solução.

A primeira etapa, de levantamento de requisitos, utilizou uma abordagem mista, partindo da observação empírica do autor sobre os desafios do setor de transporte alternativo. Tais observações foram então validadas por meio de uma pesquisa de mercado qualitativa. Para tal, foi elaborado um questionário online (detalhado no \autoref{apendice:questionario}) e aplicado junto a gestores de duas empresas do setor. As respostas, consolidadas na tabela do \autoref{apendice:resultados}, foram essenciais para a especificação formal dos requisitos que nortearam o desenvolvimento.

A segunda etapa, de desenvolvimento do protótipo, seguiu um processo iterativo alinhado a práticas de prototipagem evolutiva, onde o software é construído e refinado em ciclos contínuos \cite{sommerville2011software}. Adotou-se a estratégia \textit{Frontend-First}, que prioriza a construção da interface do usuário (UI) como guia para o desenvolvimento do sistema. Utilizando o framework Next.js e a biblioteca de componentes shadcn/ui, todas as telas da aplicação foram inicialmente desenvolvidas com dados estáticos (\textit{mockados}), permitindo a validação dos fluxos de interação antes da implementação da lógica de negócio no backend.

Por fim, a terceira etapa consistiu na avaliação do protótipo. Devido à impossibilidade de realizar testes com usuários finais, optou-se por uma verificação interna em duas frentes. A primeira foi uma \textit{Verificação de Requisitos}, na qual se analisou sistematicamente o atendimento aos requisitos funcionais e não funcionais. A segunda foi uma \textit{Avaliação Heurística}, um método de inspeção consolidado para encontrar problemas de usabilidade em interfaces \cite{Nielsen1994}. Atuando como avaliador especialista, o autor inspecionou o sistema com base nas 10 heurísticas de usabilidade de Nielsen, cujos resultados detalhados são apresentados no Capítulo 5.